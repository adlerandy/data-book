\section{introduction

Plato gave us one
of the most vivid
metaphors for the
human condition.
In the parable of
the cave
(\figref{), we are
invited to imagine
ourselves chained
to a wall, unable
to sense the
``real'' world.
All we can see are
shadows the
shadows cast
against the wall
by the light of
flickering fires
against the real
world's objects
moving and living.
We are not content
with this fate,
however, but long
to be free: to
escape our chains
and live in the outside world, in the sunlight, where we can see and interact with things as they really are.

This parable has
been interepreted
in two primary
ways: telling us
about how we learn
about what is real
(epistomolopgy),
and about the
human quest for
freedom (thus
political).
This is a book
about medical data
and its
interpretation,
and we will focus
on the first
aspect.

However, it is
interesting to
note that that two
interpretations
are not completely
independent. One
of the key
policies of
dictators is to
obscure citizens
access to the
truth. Often this
is done by
making so much
contradictory
``noise'' in the
public space,
discouraging
citizens from
feeling that
anything could be
true. In this
sense, freedom and
access to truth
are closely
related
\footnote{snyder

Returning to our
poor slaves who
see only shadows.
This is the case
of any scientist:
the things we
would really like
to know are hidden
from us, except by
the traces left in
artefacts or in
our instruments. A
historian would
love to walk on
ancient Athens'
streets and watch Plato talk; unfortunately, he can only read old texts and sculptures.
A cardiologist
would love to know
the beat-by-beat
flows and
pressures in blood
vessels over a
patient's day.
Instead, she can
read measurements
make on the body
surface when the
patient visits a
clinic.
Of course, it is
possible to put 
invasive 
instruments. For
example, Tingay et
alplace multiple
catheters in an
in-utero lamb, and
then are able to
measure the
important
physiological
changes that occur
during birth. 
But, of course,
this only tells us
about that lamb's
birth. However,
the funding for
that project was
obtained by a
promise to improve
our understanding
of human birth.
Clearly some
exprapolation is
necessary.


The remainder of
this chapter
considers the
``shadows''. In
what ways do they
pertub out access
to the real, and
in what ways are
we still able to
make reasonably
reliable
conclusions.
Figref{
illustrates the
fire, the movement
of people in the
market, and the
slaves' view of
the shadows.

- Analyze limits


A note on the terms
``real'' and
``truth''. Of
course there is
such
easy-to-describe
thing. Instead of
the movement of
horse in the
market, the full
information would
be a quantum
uncertainty model
of each atom.
We need to settle
for less, perhaps
the centre of mass
of the body. 
This quantity is 
a model, a substitude, for the real thing, which hopefully tells us enough to answer the questions we care about.
``All models are
wrong, but some
are useful'', as
the quote (often
attributed to ???)
This issue is an
important one, how
to know when our models are so wrong that they are no longer useful (see Chap {)
