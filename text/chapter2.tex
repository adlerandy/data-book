\chapter{Medical Measurements


At rest, the heart
consumes about 20\% of
the body's energy. In an
average adult, it
transports 6\,L/min (=0.1\,L/s) to a
pressure of 120\,mmHg ($\approx$ 20\,kPa) and an output power of 2\,W. 
Efficient pumping requires precise coordonation of the heart muscles and a rhythmic heart beat. 

A healthy heart is in
\emph{sinus rhythm{ an almost -- but not quite -- perfect rhythm. In fact, before the invention the of pendulum, the heart beat was the best clock available.
Variations in  heart
rhythms are caused by
heart diseases, but also
external challenges like
stress and exercise.

Perhaps the most basic
medical measurement is
the heart rate (HR), measured
in beats per minute.
It's useful both for the
absolute value and its
trend over time. 

Perhaps the least
sophisticated environment
for HR measurement is in
wilderness first aid. In
a fall outdoors there can
be a lot of pain and
lacerations. The victim's
initial HR will inevatably be high. 
The carer needs to rapidly decide whether it's serious: must we evacuate the victim, or will they get better with rest and basic care. 

This is where the \emph{trend{ is needed. HR measurements are taken every few minutes and written down (important since memory is poor during stress). If HR decreases to a resting value, this is a good sign. A constant or increasing HR is a bas sign: perhaps there is internal bleeding.

This example also illustrates a trade-off in measurement accuracy and invasiveness. Counting heart beats for a full minute allows HR to be measured to within $\pm$1\,beat/min. But this is too inconvenient normnally. Instead 15 seconds of measurement are often used, but accuracy is less $\pm$4,beat/min.

\section{Numerical medical data{

Feeling the heart rate has been medical care since our earliest records. For example, ...
Doctors were advised to feel a patient's pulse as part of assessment. This example illustrates the value of measurement -- it is not possible to calculate trends unless values are recorded.

The first to measure HR was ?? ??, who described the pulslogium.
?? worked in Pisa with Galileo who demonstrated that a pendulum can keep constant time. The pulslogium works by feeling the patient's pulse and adjusting the length of the pendulum string**
When the two are synchronized, the length is recorded.

This instrument does not directly give HR, but does give a numerical value inversely proportional to it.

It's interesting to consider whether the pulslogium is more accurate than time-based HR measurement. 
Figure ?? illustrates the range of errors possible over a short measurement windows. Given that the pressure pulse is quite narrow, this pulse provides a more precise target for the timing measurements.

The fame of ??'s work produced a revolution. It was immediately obvious how a numerical measurement of HR could provide a benefit. First, it could provide trends. Not only those described in emergency care, but also over weeks or years.

A patient re-visits a doctor and we want to know if prescribed treatment has had an effect. It is vital to refer to recorded details rather than human memory of the last visit.

Next, medical data can allow comparisons between patients and between treatments. 

It allows for sophisticated care to be more standardized and also provided by less experienced staff:
``ìf the patient's HR is above 120\,beats/min, then consider medication X''.
This effect leads to serious debate. The benefit is that a doctor can give the same care when exausted after a 24 hr shift. The disadvantage is enourages quick and absolute thinking. We will return to this point.

One further benefit is cost. We can train a less well paid staff member to make the measurements, and the well-paid doctors then save time by looking at a written result. 

Finally, and the key subject of this book, is that data can be combined and further processed in sophisticated ways. Many key clinical parameters are created from the combination of basic measurements.
 Ejection Fraction?
 FEV1

Are there downsides to measurement? Perhaps the most subtle is that what is not measured is hidden. Measurements are by their nature partial. HR records the number of beats per minute, but nothing about their rhythm, strength or character. But when we see a nicely formatted chart or graph of HR measurements, it's easy be fooled into thinking we have all the information.

In the next-section, we look at heart-rate variability.

\section{Heart-rate variability (HRV)

HR is not absolutely
constant over time. Even
a healthy heart slowly
changes rhythm over
various timescales. Early
work on HRV


\section{Discussion{

Today, we're so used to medical measurements and records, that it can be difficult to see why it wasn't obvious: surely a good doctor can see, feel (and smell) his (and it was almost always his at the time) patient. Isn't that live information better than a cold and artificial number?

This chapter introduced the earliest medical measurements and why they provide value. It turns out that there are many ways in which numbers allow better care to be provided.
Trends can be seen, care can be standardized, and cheaper people can do some of the work.

This book is more interested in the computational possibilities. Once you have a number, all sorts of mathematical tools can be applied directly. We can combine numbers and perform statistical analysis.
But again, this ease comes with a warning. Just because a mathematical tool is available, doesn't mean it's valid.

Figure {{ shows an example of a HR recording system where the value of 999 is provided to indicate a failed measurement (such a system exists!). What was X's average HR yesterday leads us to a scary value, and probably to the prescription of unnecessary medication.
