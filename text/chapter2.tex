\chapter{Medical Measurements


At rest, the heart
consumes about 20\% of
the body's energy. In an
average adult, it
transports 6\,L/min (=0.1\,L/s) to a
pressure of 120\,mmHg ($\approx$ 20\,kPa) and an output power of 2\,W. 
Efficient pumping requires precise coordonation of the heart muscles and a rhythmic heart beat. 

A healthy heart is in
\emph{sinus rhythm{ an almost -- but not quite -- perfect rhythm. In fact, before the invention the of pendulum, the heart beat was the best clock available.
Variations in  heart
rhythms are caused by
heart diseases, but also
external challenges like
stress and exercise.

Perhaps the most basic
medical measurement is
the heart rate (HR), measured
in beats per minute.
It's useful both for the
absolute value and its
trend over time. 

Perhaps the least
sophisticated environment
for HR measurement is in
wilderness first aid. In
a fall outdoors there can
be a lot of pain and
lacerations. The victim's
initial HR will inevatably be high. 
The carer needs to rapidly decide whether it's serious: must we evacuate the victim, or will they get better with rest and basic care. 

This is where the \emph{trend{ is needed. HR measurements are taken every few minutes and written down (important since memory is poor during stress). If HR decreases to a resting value, this is a good sign. A constant or increasing HR is a bas sign: perhaps there is internal bleeding.

This example also illustrates a trade-off in measurement accuracy and invasiveness. Counting heart beats for a full minute allows HR to be measured to within $\pm$1\,beat/min. But this is too inconvenient normnally. Instead 15 seconds of measurement are often used, but accuracy is less $\pm$4,beat/min.

 or
estimated from a shorter
interval such as 15 or 30 
